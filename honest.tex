% ===== 至诚 =====
\section{至诚}

% 配色与背景
\definecolor{trustwhite}{RGB}{200,40,50}  % 亮红色
\definecolor{textdarkgray}{RGB}{40,40,40}

\setbeamertemplate{background canvas}{
  \begin{tikzpicture}[remember picture, overlay]
    \node[anchor=south west, at=(current page.south west)] {
      \includegraphics[width=\paperwidth,height=\paperheight]{bg-honest.png}
    };
  \end{tikzpicture}
}

% ---- 封面页 ----
\begin{frame}[plain]
  \centering
  \vspace{3.5cm}
  {\color{trustwhite}\Huge\bfseries 至诚}
\end{frame}

% ---- 没人带,不等于没人配合 ----
\begin{frame}{\color{trustwhite}没人带,不等于没人配合}
\justifying
{\color{textdarkgray}
很多同学说:“我们队没人带。”\\
但更深层的问题是:\\
我们没进入协作角色。

\vspace{1em}
至诚,不是口号,\\
而是你愿不愿意成为那个\textbf{让团队放心的人}。
}
\end{frame}

% ---- 至诚,是团队中的“可托之人” ----
\begin{frame}{\color{trustwhite}至诚,是团队中的“可托之人”}
\justifying
{\color{textdarkgray}
“靠谱”这个词,不只是性格标签。\\

\vspace{0.5em}
它意味着:
\begin{itemize}
  \item \textbf{状态透明}:你在哪一块、进展如何;
  \item \textbf{反馈清晰}:不沉默、不甩锅、敢说不会;
  \item \textbf{承诺守信}:说到就做到,做不到提前说。
\end{itemize}

\vspace{0.5em}
\textbf{至诚},是信任机制中的人格担保。
}
\end{frame}

% ---- 三人团队结构 ----
\begin{frame}{\color{trustwhite}三人团队结构:领导者、执行者、调整者}
\justifying
{\color{textdarkgray}
一个小团队,最稳的结构是三角:

\vspace{0.8em}
\begin{itemize}
  \item \textbf{领导者(Leader)}:宏观统筹,拆解任务,把控节奏;
  \item \textbf{执行者(Doer)}:专注产出,代码建模,主线推进;
  \item \textbf{调整者(Mediator)}:查漏补缺,调参优化,文档补强。
\end{itemize}

\vspace{0.8em}
三人互补,共同封闭建模任务的“方向—推进—容错”结构。
}
\end{frame}

% ---- 三类人的定位与协作 ----
\begin{frame}{\color{trustwhite}三类人的定位与协作}
\justifying
{\color{textdarkgray}
\textbf{领导者}:不在于“技术最强”,而在于能组织任务有章法。\\
\textbf{执行者}:不是独狼冲刺,而是随时汇报、保障主线可控。\\
\textbf{调整者}:不是打杂,而是补关键缺口,是团队的“护法”。

\vspace{0.5em}
三者间最重要的,是节奏同步与边界明确。
}
\end{frame}

% ---- 团队合作三准则 ----
\begin{frame}{\color{trustwhite}团队合作的三项准则}
\justifying
{\color{textdarkgray}
\begin{enumerate}
  \item \textbf{沟通优先}:遇到问题先沟通,不要默默死扛;
  \item \textbf{状态透明}:不要让队友猜你在干嘛;
  \item \textbf{承诺守信}:每次任务都是信用试炼。
\end{enumerate}

\vspace{0.5em}
\centering
\textbf{协作,不是分工,而是共同承担责任。}
}
\end{frame}

% ---- 至诚,是数据与模型的态度 ----
\begin{frame}{\color{trustwhite}数据诚实与模型众理}
\justifying
{\color{textdarkgray}
模型源于数据,分析基于逻辑,态度必须真实。

\vspace{1em}
在团队中,每一个代码、图表、结果、解释,\\
都是别人要相信的内容,\\
你能不能负责地说:“这是可以交的”。

\vspace{1em}
\textbf{至诚,是你对数据的诚,是你对团队的信。}
}
\end{frame}
