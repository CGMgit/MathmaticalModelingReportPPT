% ===== 笃行 =====
\section{笃行}

% 配色与背景
\definecolor{steelwhite}{RGB}{255,255,255}  % 比 white 更柔和一些
\definecolor{textblack}{RGB}{255,255,255}

\setbeamertemplate{background canvas}{
  \begin{tikzpicture}[remember picture, overlay]
    \node[anchor=south west, at=(current page.south west)] {
      \includegraphics[width=\paperwidth,height=\paperheight]{bg-action.png}
    };
  \end{tikzpicture}
}

% ---- 封面页 ----
\begin{frame}[plain]
  \centering
  \vspace{3.5cm}
  {\color{steelwhite}\Huge\bfseries 笃行}
\end{frame}

% ---- 做不了,是因为落地不了 ----
\begin{frame}{\color{steelwhite}做不了,是因为落地不了}
\justifying
{\color{textblack}
许多同学的痛点不是“不理解”,而是:

\vspace{0.5em}
\begin{itemize}
  \item 有想法,却无法实现;
  \item 有公式,却跑不出代码;
  \item 有模型,却生成不了结果。
\end{itemize}

\vspace{0.5em}
“笃行”是让思想落地的能力,是将理论转化为实践的路径。
}
\end{frame}

% ---- 工程落地的根:计算机能力 ----
\begin{frame}{\color{steelwhite}工程落地的根:计算机能力}
\justifying
{\color{textblack}
模型落地,不靠灵感,靠系统:

\vspace{0.5em}
\begin{itemize}
  \item 数据结构与数值算法;
  \item 文件组织与模块结构;
  \item 调试分析与复现逻辑;
  \item 工具链路与任务自动化。
\end{itemize}

\vspace{0.5em}
计算机能力,不是锦上添花,而是建模的地基。
}
\end{frame}

% ---- 四维能力闭环 ----
\begin{frame}{\color{steelwhite}从“做不了”到“跑起来”:四维能力闭环}
\justifying
{\color{textblack}
\textbf{1. 建模实现力}:能将理论表达为代码。\\
\textbf{2. 系统组织力}:有结构的项目架构与版本管理。\\
\textbf{3. 复现与调试力}:错误可查,模型可控。\\
\textbf{4. 工具融合力}:将计算、排版、展示一体化。
}
\end{frame}

% ---- 工具地图 ----
\begin{frame}{\color{steelwhite}工具为骨,系统为脉}
\justifying
{\color{textblack}
常用建模工具推荐(部分示例):

\begin{itemize}
  \item \textbf{代码工具:} GitHub, VS Code, Jupyter
  \item \textbf{建模语言:} Python (numpy, scipy, sklearn), MATLAB
  \item \textbf{排版展示:} Overleaf, LaTeX, TikZ
  \item \textbf{协作工具:} Trello, Notion, Git
  \item \textbf{AI助手:} ChatGPT, GitHub Copilot
\end{itemize}

\vspace{0.5em}
\centering
善用工具,是执行力的最小单位。
}
\end{frame}

% ---- AI × 动手公式页 ----
\begin{frame}{\color{steelwhite}AI时代,动手是指数维度的能力}
\justifying
{\color{textblack}
\[
\text{建模执行力} \approx \text{动手能力} \times \exp(\text{AI 助力指数})
\]

\vspace{0.5em}
\begin{itemize}
  \item 不会动手 → AI 也帮不了你;
  \item 会动手 → AI 成为你的超能力;
\end{itemize}

\vspace{0.5em}
AI 是加速器,动手是起跑线。
}
\end{frame}

% ---- 精神收束 ----
\begin{frame}{\color{steelwhite}动手,是唯一的出路}
\justifying
{\color{textblack}
在 AI 帮你跑图之前,\\
你要先学会自己跑一行代码;\\
在模型结果出来之前,\\
你要先把路径搭起来。

\vspace{1em}
\centering
\textbf{\color{steelwhite}“笃行”,是让结果真实发生的唯一路径。}
}
\end{frame}
