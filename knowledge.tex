% ===== 博学 =====
\section{博学}

% 定义专属颜色(高对比冷光蓝)
\definecolor{techwhite}{RGB}{0,0,0}

% 背景图设定
\setbeamertemplate{background canvas}{
  \begin{tikzpicture}[remember picture, overlay]
    \node[anchor=south west, at=(current page.south west)] {
      \includegraphics[width=\paperwidth,height=\paperheight]{bg-knowledge.png}
    };
  \end{tikzpicture}
}

% ---- 封面页 ----
\begin{frame}[plain]
  \centering
  \vspace{3.5cm}
  {\color{techwhite}\Huge\bfseries 博学}
\end{frame}

% ---- 本质引入 ----
\begin{frame}{\color{techwhite}建模,是一种科学思维}
\justifying
{\color{techwhite}
建模不是做题,而是\textbf{组织知识、构建系统、生成解释}的过程。

\vspace{0.5em}
这意味着它和科学研究类似——

我们不是找到一个正确答案,而是构建一个逻辑自洽的解释系统。
}
\end{frame}

% ---- 演绎与归纳 ----
\begin{frame}{\color{techwhite}科学推理的两条主线}
\justifying
{\color{techwhite}
\textbf{演绎(Deduction):}
\begin{itemize}
  \item 从已有理论出发,推导模型与结果;
  \item 建模时常用于方程建构、控制逻辑。
\end{itemize}

\textbf{归纳(Induction):}
\begin{itemize}
  \item 从数据中提取规律,再构建解释;
  \item 建模时常用于拟合预测、模式识别。
\end{itemize}

\vspace{0.5em}
比赛多数问题,是\textbf{以演绎为主导},在范式中选择与优化。
}
\end{frame}

% ---- 知识范式与从有到好 ----
\begin{frame}{\color{techwhite}从有到好:范式驱动的推理过程}
\justifying
{\color{techwhite}
很多同学以为建模是“从零开始”。

其实——
\begin{itemize}
  \item 题目有背景,方法有历史;
  \item 我们不是在造房子,而是在已有图纸上优化结构;
  \item 本质上是“从有到好”的演绎过程。
\end{itemize}

\vspace{0.5em}
\textbf{博学},是知道在哪类问题中用什么范式,怎么变形与组合。
}
\end{frame}

% ---- 写作链条 ----
\begin{frame}{\color{techwhite}写作,是博学的外化过程}
\justifying
{\color{techwhite}
建模论文的结构:
\begin{itemize}
  \item 问题定义 → 模型构建 → 推理求解 → 实验验证 → 分析结论;
\end{itemize}

但写作中最大的问题:
\begin{itemize}
  \item 模型从哪里来?
  \item 结构为什么这样?
  \item 数据怎么解释?
\end{itemize}

\textbf{博学}决定你是否知道:你到底在说什么、推了什么、怎么合理。
}
\end{frame}

% ---- 学不会怎么办?----
\begin{frame}{\color{techwhite}知识太多,我们该怎么学?}
\justifying
{\color{techwhite}
学生常问:

\vspace{0.5em}
\begin{itemize}
  \item 知识太杂,学不动;
  \item 工具太多,记不住;
  \item 题型太广,不会归纳。
\end{itemize}

\vspace{0.5em}
\textbf{博学的第一步,不是学全部,而是建立“知识系统”}

\centering
\textbf{从问题出发,划分类别,掌握经典模型。}
}
\end{frame}

% ---- 模型系统六分类 ----
\begin{frame}{\color{techwhite}六大问题类型,对应六类模型系统}
\justifying
{\color{techwhite}
大多数建模题都可归入六种任务场景:

\vspace{0.5em}
\begin{itemize}
  \item \textbf{预测类:} 回归、时间序列、马尔科夫链;
  \item \textbf{评价类:} AHP、TOPSIS、聚类分析;
  \item \textbf{优化类:} 线性/整数规划、动态规划;
  \item \textbf{自动控制类:} MPC、强化学习、状态空间;
  \item \textbf{传播与调度类:} ODE模型、图论调度、遗传算法;
  \item \textbf{分类识别类:} 决策树、SVM、KNN、逻辑回归。
\end{itemize}
}
\end{frame}
