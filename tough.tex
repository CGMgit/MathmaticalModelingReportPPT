% ===== 至坚 =====
\section{至坚}

% ==== 配色定义(高可读性)====
\definecolor{deepnavy}{RGB}{30,50,100}          % 标题色:沉稳深蓝
\definecolor{darkcharcoal}{RGB}{255,255,255}       % 正文色:白色
\definecolor{emphasisred}{RGB}{150,20,20}       % 强调词:暗红强调
\definecolor{whitebgtext}{RGB}{255,255,255}     % 如需标题落在深色背景,可切换白色

% ==== 背景图设置 ====
\setbeamertemplate{background canvas}{
  \begin{tikzpicture}[remember picture, overlay]
    \node[anchor=south west, at=(current page.south west)] {
      \includegraphics[width=\paperwidth,height=\paperheight]{bg-tough.png}
    };
  \end{tikzpicture}
}

% ---- 封面页 ----
\begin{frame}[plain]
  \centering
  \vspace{3.5cm}
  {\color{deepnavy}\Huge\bfseries 至坚}
\end{frame}

% ---- 怕失败,是因为没准备好心态 ----
\begin{frame}{\color{deepnavy}怕失败,是因为没准备好心态}
\justifying
{\color{darkcharcoal}
很多人不是输在能力上,而是输在\textbf{心理预设失败}。

\vspace{1em}
他们在开始前,就已经为失败找好了借口;\\
他们在卡住时,就习惯了抱怨、推责、逃避。

\vspace{1em}
\textbf{\color{emphasisred}至坚},是重新夺回面对困难的主动权。
}
\end{frame}

% ---- 至坚,是你失败后依然愿意前进 ----
\begin{frame}{\color{deepnavy}至坚,是你失败后依然愿意前进}
\justifying
{\color{darkcharcoal}
\begin{itemize}
  \item 成功不是没有失败,而是失败之后还能前行;
  \item 不动声色地重来,是一种无声的英雄主义;
  \item 你可以慢,但不能停;你可以痛,但不能退。
\end{itemize}

\vspace{0.5em}
\textbf{\color{emphasisred}迎难而上},不是冲动,是训练过的意志。
}
\end{frame}

% ---- 坚韧,不是情绪上头,而是行为持续 ----
\begin{frame}{\color{deepnavy}坚韧,不是情绪上头,而是行为持续}
\justifying
{\color{darkcharcoal}
建模不是看谁激动,而是看谁能稳住:

\vspace{0.8em}
\begin{itemize}
  \item \textbf{不抱怨:} 冷静面对 bug、报错、图跑崩;
  \item \textbf{注重细节:} 每个图都调过字,每个公式都对齐;
  \item \textbf{逼着不挠:} 再跑一次、再查一遍、再拼一图。
\end{itemize}

\vspace{0.5em}
\textbf{\color{emphasisred}至坚}不是靠热血,而是靠心血。
}
\end{frame}

% ---- 三种强者心态 ----
\begin{frame}{\color{deepnavy}三种强者心态}
\justifying
{\color{darkcharcoal}
\begin{enumerate}
  \item \textbf{静心:} 失败后不骂人,卡住时不内耗;
  \item \textbf{细节:} 不将就每一个截图、不妥协每一行注释;
  \item \textbf{心血:} 不是用时间堆砌,而是投入情绪与专注。
\end{enumerate}

\vspace{0.5em}
\textbf{坚韧,是“把事情做好”的决心,而非“做完就行”的侥幸。}
}
\end{frame}

% ---- 建模最后一天 ----
\begin{frame}{\color{deepnavy}建模最后一天:撑住的人,才有故事}
\justifying
{\color{darkcharcoal}
每场建模的最后一天,都会发生这几件事:

\vspace{0.5em}
\begin{itemize}
  \item 一个队友通宵修图,输出封面;
  \item 一个队友边流泪边写分析;
  \item 一个队伍,把不可能变成了“交得上”。
\end{itemize}

\vspace{0.5em}
\textbf{\color{emphasisred}至坚}不是不怕崩,而是明知道会崩,依然坚持撑完。
}
\end{frame}

% ---- 收束:至坚,是你对自己的兑现 ----
\begin{frame}{\color{deepnavy}至坚,是你对自己的兑现}
\justifying
{\color{darkcharcoal}
\begin{center}
你可以不完美,\\
但你不能不努力走完。

\vspace{1em}
你可以不是最强的那一个,\\
但你要做\textbf{\color{emphasisred}最后不退的那一个}。
\end{center}

\vspace{1em}
\centering
\textbf{\color{deepnavy}至坚,是你走完这条建模之路的最后一公里。}
}
\end{frame}
